\chapter{Objektově orientované programování (OOP)}

Naši cestu za objektově orientovaným programováním započneme citátem Alana Kaye, spoluautora jednoho
z prvních objektově orientovaných programovacích jazyků jménem Smalltalk:
\vskip0.5cm
\noindent
``Actually I made up the term `object-oriented', and I can tell
you I did not have C++ in mind.''
\vskip0.3cm
\noindent
The Computer Revolution hasn’t happend yet -- Keynote, OOPSLA 1997

\vskip0.5cm

V této kapitole si představíme objektově orientované programovaní (OOP) v jazyce Python.
Objektově orientované programování není v programování novým pojmem, stále však je přechod
od procedurálního programování k objektovému někdy obtížný. Někdy může být na překážku statická typovost
některých jazyků. My si ukážeme, že je OOP velmi jednoduchý a příjemný způsob programování. Nepodlehněme
však představě, že je OOP vše řešící technika.


\section{Co jsou objekty}

Z dosavadní práce v jazyce by nám již mělo být dostatečně jasné, že vše v jazyce Python je objektem.
Např. listy obsahující prvky jsou objekty a také elementy v nich jsou samy objekty. I přes tuto skutečnost
nám není jazykem objektový model nějak extrémně vnucován. Můžeme se de facto setkat pouze s klasickou
tečkovou notací pro volání metod objektů, např. \kod{lst.append(1)} pro přidání objektu typu
\kod{int} do objektu typu \kod{list} (proměnná \kod{lst}). Tak, jak jsme zvyklí z jazyka C++, i v Pythonu
můžeme volání metod nebo atributů zřetězit, např. \\ \kod{instance\_tridy.metoda().atribut}. Tečková notace
také nerozlišuje mezi voláním metod nebo přístupu k atributům. Toto je rozlišeno pouze kulatými
závorkami u volání metod.


\section{Motivace k zavedení objektů}

Pokud jste programovali v jazyce C, většina programů nejprve využívala funkce, které měly parametry.
Jak program nabýval na složitosti, parametrů u funkcí mohlo přibývat. Takovýto proces je více, či méně
nevyhnutelný. Pokud jste pracovali dále, možná jste se rozhodli, že si vytvoříte jednoduchou strukturu,
do které si uložíte hodnoty parametrů. Tímto způsobem jste dosáhli toho, že jste nahradili mnoho parametrů
jedním. Díky tomu jste byli schopni centralizovat data. Kolem těchto dat jste vytvořili funkce, které
např. přijímaly danou strukturu jako svůj argument a vracely vám hodnotu, popřípadě upravovaly danou
strukturu.

Toto je již docela zajímavý způsob, jak si zjednodušit práci s daty v jazyce C. Problém však nastává
v situaci, kdy bychom chtěli mít jednu funkci (myšleno z pohledu API), která by se chovala pouze podle
toho, jakou strukturu jí pošleme jako argument. V takovém případě musíme nejprve v takové funkci zjistit
o jakou strukturu se jedná a podle toho patřičně reagovat (spustit patřičný kód). Ukázka takového přístupu
je se zjednodušeními ukázána níže (poznámka: nejedná se o konkrétní programovací jazyk).

\begin{python}
struct {
    name = '';
} Dog;

struct {
    name = '';
} Cat;

make_sound( struct ) {
    if type( struct ) == Dog {
        print struct.name + 'Haf!';
    }
    else if type( struct ) == Cat {
        print struct.name + 'Minau!';
    }
}

struct dog = Dog( "Lassie" );
struct cat = Cat( "Tom" );

make_sound( dog ); // 'Lassie: Haf!'
make_sound( cat ); // 'Tom: Minau!'
\end{python}

V příkladu jsme vytvořili funkci \kod{make\_sound}, která přijímá jeden argument v podobě struktury.
Danou strukturou může být buď struktura \kod{Dog} nebo \kod{Cat}. Evidentně po funkci \kod{make\_sound}
chceme, aby vytiskla typický zvuk pro dané zvíře. Toho docílíme tak, že se zeptáme, jakého typu je
argument a podle něj již patřičně zareagujeme. Tento přístup není obecně špatný, avšak při větším množství
struktur se stává značně nepřehledným. Výše popsaný příklad se také snaží o to, aby jedna funkce byla
schopna reagovat na více datových typů. Zde bychom již měli vidět, že to, co po funkci požadujeme, je
de facto polymorfismus (samozřejmě ještě stále zjednodušený).


\section{Objektový přístup k motivačnímu problému}

Z předchozích kurzů víte, že OOP se skládá ze tří pilířů, kterými jsou:
polymorfismus, dědičnost a zapouzdření.
Věnujme se nyní primárně nejpodstatnějšímu pojmu, kterým je polymorfismus. V předchozím příkladu 
chceme funkčnost, kterou bychom mohli vágně popsat následovně: ,,Jakémukoli objektu pošleme zprávu
a dostaneme adekvátní odpověď''. Toto je hodně obecný koncept, který je třeba si přiblížit. Zasílání
zpráv je v moderním konceptu OOP jen voláním metod objektu, který je nám již důvěrně znám pomocí
tečkové notace. OOP za nás zajistí vše ostatní. Ukažme si tedy předchozí příklad v jazyce Python.

\begin{python}
class Dog(object):
    def __init__(self, name):
        self.name = name

    def make_sound(self):
        print("{0}: Haf!".format(self.name))


class Cat(object):
    def __init__(self, name):
        self.name = name

    def make_sound(self):
        print("{0}: Minau!".format(self.name))

animals = [ Dog("Lassie"), Cat("Tom") ]

for animal in animals:
     animal.make_sound()

# Lassie: Haf!
# Tom: Minau!
\end{python}

Na chvíli ještě pomiňme syntaxi pro tvorbu tříd. Vidíme však, že oproti předchozímu příkladu je funkce
\kod{make\_sound} přítomna v obou třídách a je implementována podle toho, ke které třídě patří.

Dále vidíme, že jsme si vytvořili list \kod{animals}, který obsahuje instance tříd \kod{Dog} a \kod{Cat}.
Nakonec projdeme tento list a na každém jeho prvku zavoláme metodu \kod{make\_sound}.

Vidíme, že při volání metody \kod{make\_sound} již do ní neposíláme žádný parametr (jako v předchozím případě
strukturu, \kod{self} teď nepočítejme).
Posun od procedurálního programování k OOP je tedy již docela markantní. V cyklu \kod{for}
pak voláme na jednotlivých instancích metodu 
\\
\kod{make\_sound} bez toho, aniž bychom primárně věděli, o který
typ se jedná. Toto je také nejdůležitějším prvkem OOP -- volání metod na objektech a nestarání se o jejich
vnitřní implementaci.

Dále si v příkladu všimněme, že list \kod{animals} obsahuje instance tříd \kod{Dog} a \kod{Cat}. Na rozdíl
od jazyka C++ nebo Java jsme však nemuseli vytvářet společnou nadtřídu, popřípadě interface. Toto je dáno
dynamickou typovostí jazyka, se kterou jsme se setkali již dříve. List mohl obsahovat integery, floaty,
stringy, apod. Není tedy problém, aby obsahoval i instance námi vytvořených tříd. V dynamicky typovaných
jazycích také můžeme volat jakoukoli metodu na jakémkoli objektu. Pokud však objekt neumí na metodu reagovat
(metoda není vůbec implementována), je vyhozena výjimka, kterou musíme zpracovat.


\section{Třídy a instance}

Jazyk Python pracuje s objektovým modelem podobně jako C++ nebo Java. Třídy jsou obecným předpisem pro
popis objektů světa. Instance jsou pak konkrétními objekty, které mají svou třídu jako předpis.
Toto bychom si mohli představit i tak, že automobilka produkuje auta podle předpisu (výrobního postupu).
Automobilka má tedy sadu tříd, podle kterých se ve výrobě vytvářejí automobily, které jsou již instancemi
tříd.


\section{Definice třídy a konstrukce objektu}

Rozeberme si na příkladu vytvoření objektu, speciálně syntaxi konstruktoru, která je lehce odlišná
od C++ nebo Javy.

\begin{python}
class Battlestar(object):
    
    def __init__(self, name, commander): # constructor
        self.name = name                 # instance variable
        self.commander = commander

    def identify(self):                  # method
        return 'This is Battlestar {0}, \
                commanded by {1}.'.format(self.name, \
                self.commander)

galactica = Battlestar('Galactica', 'Bill Adama')
pegasus = Battlestar('Pegasus', 'Helena Cain')

print(galactica.identify())
# This is Battlestar Galactica, commanded by Bill Adama.

print(pegasus.identify())
# This is Battlestar Pegasus, commanded by Helena Cain.
\end{python}

    Definice přídy je uvozena klíčovým slovem \kod{class}\index{class} následována jménem.
Pokud chce definovaná třída
dědit z jiných tříd, je toto uvedeno v kulatých závorkách. Naše třídy budou vždy dědit z třídy reprezentující
objekt (\kod{object}\index{object}).

Konstruktor třídy je vždy metoda s názvem \kod{\_\_init\_\_}\index{\_\_init\_\_},
která přijímá jako svůj první argument proměnnou,
která je tradičně nazývána \kod{self}\index{self}.
Pak následuje seznam argumentů, které chceme konstruktoru předat.
Proměnná \kod{self} nám reprezentuje instanci podobně, jako je tomu v jazyce C++ nebo Java slovem \kod{this}.
Instanční proměnné jsou vytvořeny tak, že je přidáme do instance pomocí tečkové notace. Dostatečný příklad
je uveden v konstruktoru.
Pokud se nám zdá syntaxe metod nepříjemná v tom smyslu, že musíme uvádět proměnnou \kod{self}, tak se
zamysleme nad tím, kde se vezme v C++ nebo Javě proměnná \kod{this}. Tato proměnná je ve skutečnosti
do zdrojového kódu ,,vpašována'' překladačem daného jazyka.

Metody třídy obdobně přijímají první argument v podobě proměnné \kod{self}. Přístup k instančním proměnným
je možný opět pouze s použitím oné proměnné.

Vytvoření objektu je velmi podobné jako v jiných jazycích, které známe. Jediným rozdílem je to, že
nepoužíváme klíčového slova \kod{new}\index{new}. Jazyk Python (stejně jako téměř ve všech případech Java)
totiž na rozdíl od jazyka C++ neumožňuje
vytvořit instanci objektu na zásobníku, ale pouze na haldě.

\section{Třídní proměnné}

Instanční proměnné by pro nás již neměly být problémem.
Každá instance třídy si udržuje své vlastní hodnoty v instančních proměnný.
Existují však i jiné proměnné, které nazýváme třídní proměnné a ty jsou pro všechny instance společné.
Tyto proměnné jsou dosti podobné globálním proměnným, na které jsme zvyklí ať už z jazyka C nebo z Pythonu.
Otázkou však je, jak takové proměnné v Pythonu deklarovat a využít.

Představme si, že bychom chtěli vytvořit internetový vyhledávač. Pomiňme nyní relativně velkou složitost
tohoto problému a soustřeďme se na to, jak bychom reprezentovali jednotlivé webové stránky v našem systému.

\begin{python}
class Document(object):
    
    def __init__(self, content):
        self.content = content

    def index(self, db):
        pass
\end{python}

Třída \kod{Document} tedy reprezentuje stránku pomocí instanční proměnné \kod{content}. Metoda \kod{index}
pak danou stránku zaindexuje do databáze \kod{db}. Její implementaci však nebudeme řešit.
Dříve nebo později bychom však chtěli vědět, kolik dokumentů v našem systému je. Jednou z možností je zeptat
se databáze. Druhou možností je pak využít třídní proměnné tak, jak je ukázáno na následujícím příkladu.

\begin{python}
class Document(object):
    
    no_of_documents = 0

    def __init__(self, content):
        self.content = content
        Document.no_of_documents += 1

    def index(self, db):
        pass

d1 = Document("Text")
print(d1.no_of_documents) # 1

d2 = Document("Another text")
print(d1.no_of_documents) # 2
print(d2.no_of_documents) # 2

print(Document.no_of_documents) # 2
\end{python}

Pokaždé, kdy vytvoříme novou instanci dokumentu, je třídní proměnná \kod{no\_of\_documents} inkrementována.
Na počet dokumentů se tak můžeme zeptat jakékoli instance, protože je třídní proměnná sdílená.
K třídní proměnné přistupujeme přes jméno třídy. Nemůžeme tak použít notaci přes \kod{self},
poněvadž \kod{self} referuje na instanci a nikoli na třídu.
Nakonec k třídní proměnné můžeme přistoupit i tak, že se zeptáme přímo třídy samotné.
V naší ukázce je to poslední řádek kódu.

\section{Statické metody}

Již dříve jsme si řekli, že v OOP reprezentujeme objekty reálného světa. Jejich atributy jsou instanční
proměnné a pro jejich manipulaci máme metody, které jsou svázány s objekty, což je zajištěno tím, že
jsou definovány u dané třídy. Někdy však potřebujeme to, abychom byli schopni volat metodu aniž bychom
měli k dispozici konkrétní instanci. Chceme však využít toho, že metoda má implementovánu funkcionalitu,
která se nějakým způsobem váže k dané třídě. Využijme tedy opět našeho příkladu pro vyhledávač.
Nyní bychom chtěli vědět průměrnou délku stránek, které indexujeme.

\begin{python}
class Document(object):
    
    no_of_documents = 0
    total_length = 0

    def __init__(self, content):
        self.content = content
        Document.no_of_documents += 1
        Document.total_length += len(self.content)

    def index(self, db):
        pass

    @staticmethod
    def get_average_length():
        return Document.total_length / Document.no_of_documents

d1 = Document("Text")
d2 = Document("Another text")

print(Document.get_average_length()) # 8
print(d1.get_average_length()) # 8
\end{python}

Na první pohled vidíme oproti běžné metodě dva rozdíly. Prvním je to, že metoda je uvozena
dekorátorem\index{dekorátor} \kod{@staticmethod}\index{@staticmethod}. Tento dekorátor nám nezajistí ani tak
to, že bude metoda statická, jako spíše to, že ji jako statickou bude možno volat na intanci, což je
v jiných jazycích problematické. Toto je ukázáno na posledním řádku příkladu.
\textbf{Podstatným rozdílem} je však to, že metoda nemá jako svůj první
argument proměnnou \kod{self}. Je tedy explicitně řečeno, že metoda nemůže být volána na objektu (již
víme, že objekt je předán jako argument metody v podobě proměnné \kod{self}). Dále je samozřejmé, že
statická metoda nemůže přistupovat k instančním proměnný. Logiku tohoto tvrzení ponechme na čtenáři.
\\
\\
\noindent
{\textbf{Cvičení:}}
Zkuste odstranit dekorátor a spustit příklad, případně proveďte další úpravy.


\section{Vlastnosti (Properties)\index{properties}}

Další vlastností OOP je zapouzdření\index{zapouzdření}. Jazyk Python nepodporuje skrývání instančních
proměnný tak, jak to například dovoluje Java pomocí modifikátorů \kod{private} nebo \kod{protected}.
Naproti tomu je v Pythonu sada doporučení, jak se k zapouzdření zachovat. S takovýmto přístupem se můžeme
setkat i u jiných jazyků. Vzpomeňme si, že je třeba při ředění kyseliny nalévat kyselinu do vody
a nikoli opačně. Toto je také doporučení a nikdo nám nebrání to udělat obráceně.
Následky však mohou být dosti nepříjemné.

Přímá manipulace s atributy třídy je sice možná, ale pokud nejsme přímými autory těchto tříd, je dosti
pravděpodobné, že takovouto neopatrnou manipulací způsobíme nekonzistenci dat. Standardní způsob
manipulace s atributy tříd je přes metody, které jsou přímo určeny k zápisu nebo čtení hodnot těchto
atributů. Způsob práce je podobný jako v jazyce C++ nebo Java. Ukažme si tedy příklad.

\begin{python}
class Document(object):
    
    no_of_documents = 0
    total_length = 0

    def __init__(self, content):
        self.content = content
        Document.no_of_documents += 1
        Document.total_length += len(self.content)

    def set_content(self, content):
        Document.total_length -= len(self.content)
        self.content = content
        Document.total_length += len(self.content)

    def get_content(self):
        return self.content

    def index(self, db):
        pass

    @staticmethod
    def get_average_length():
        return Document.total_length / Document.no_of_documents

d1 = Document("Text")
d2 = Document("Another text")

print(Document.get_average_length()) # 8

d2.set_content("This is a bit longer text")

print(Document.get_average_length()) # 14
\end{python}

V metodě \kod{set\_content} do instanční proměnné \kod{content} uložíme nový obsah stránky.
Je však také nutné přepočítat průměrnou délku dokumentů, které máme v našem systému. Toto nám metoda
\kod{set\_content} zajistí. Pokud bychom však přímo manipulovali s obsahem proměnné \kod{content},
toto by zajisté zajištěno nebylo. Metoda \kod{get\_content} pak jen vrací obsah proměnné \kod{content}.
Na tomto příkladu bychom měli jasně vidět proč je vhodné používat tzv. setry a getry.
\\
\\
\noindent
{\textbf{Cvičení:}}
Jako cvičení si zkuste přímou manipulaci s proměnnými a zjistěte, kdy nastává nekonzistence dat a jak
se projevuje.
\\
\\
V Pythonu však můžeme ještě využít tzv. \textbf{properties}\index{properties}. Ty se používají u atributů, které jsou pro
nás zajímavé a vytvářejí iluzi přímé manipulace. Následující ukázka ilustruje jejich použití.

\begin{python}
class Document(object):
    
    no_of_documents = 0
    total_length = 0

    def __init__(self, content):
        self._content = content
        Document.no_of_documents += 1
        Document.total_length += len(self.content)

    def set_content(self, content):
        Document.total_length -= len(self._content)
        self._content = content
        Document.total_length += len(self._content)

    def get_content(self):
        return self._content

    def index(self, db):
        pass

    @staticmethod
    def get_average_length():
        return Document.total_length / Document.no_of_documents

    content = property(get_content, set_content)

d1 = Document("Text")
d2 = Document("Another text")

print(Document.get_average_length()) # 8

d2.content = "This is a bit longer text"

print(d2.content) # "This is a bit longer text"
print(Document.get_average_length()) # 14
\end{python}

Naši původní proměnnou \kod{content} jsme přejmenovali na \kod{\_content}, což výsledek nijak neovlivňuje
(toto jsme museli udělat proto, že jinak by docházelo ke kolizi jmen mezi property a proměnnou, která by
vyústila v nekonečnou rekurzi, o čemž se můžete přesvědčit, pokud si příklad náležitě upravíte).
Nakonec jsme pomocí funkce \kod{property} řekli, že chceme zpřístupnit data pod názvem \kod{content},
která se čtou a nastavují pomocí
funkcí \kod{get\_content} a \kod{set\_content}.
Ve výsledku můžeme jakoby přímo zapisovat do proměnné \kod{content} a přitom je zajištěna konzistence dat.
Další výhodou je to, že pokud později změníme set nebo get metody, kód který používá properties se nemusí
měnit. Tím máme pěkně zajištěnu dopřednou kompatibilitu.


\section{Operátory}

V průběhu naší práce jsme se setkali s operátory a nepřišlo nám to nijak složité. Např. jsme mohli sčítat
čísla pomocí operátoru \kod{+} nebo spojovat řetězce tím samým operátorem. Problémem těchto operátorů však
je to, že je tvůrci naučili pracovat pouze s vestavěnými typy, např. \kod{int}, \kod{str}, apod.

Pokud vytvoříme svou vlastní třídu, ta se stává novým datovým typem (toto obecně nazýváme abstraktní datové
typy). Problém však je, že nyní standardní operátory nemohou tušit, jak s našimi vlastními typy pracovat.
A právě k tomu nám slouží přetěžování operátorů. Seznam operátorů, které můžeme přetížit je relativně dlouhý,
a proto se odkážeme na dokumentaci \cite{python-operator-overloading}. Nám aktuálně postačí to, že operátory
jsou v třídách realizovány jako metody, které vždy začínají a končí znaky dvou podtržítek \kod{\_\_}.
Mezi nimi je pak jméno operátoru. Pozorný čtenář již tuší, že konstruktor je také jen operátorem.

Jako příklad přetížení operátoru si ukážeme, jak naimplementovat třídu \kod{Vector}, která reprezentuje
vektor libovolné délky. Budeme chtít sčítat dva různé vektory a také budeme chtít tisknout jejich obsah
pomocí příkazu \kod{print}.

\begin{python}
class Vector(object):
    def __init__(self, vec):
        #warning: vec will be tuple, immutable
        self.vec = vec

    def __add__(self, other):
        vec = []
        for no, item in enumerate(self.vec):
            vec.append(item + other.vec[no])

        return Vector(tuple(vec))

    def __str__(self):
        #content = ", ".join([str(i) for i in self.vec])
        #return "({})".format(content)
        return str(self.vec)

v1 = Vector((1, 2, 3, 4))
print(v1)

v2 = Vector((5, 6, 7, 8))
print(v2)

v3 = v1 + v2
print(v3)
\end{python}

V ukázce vidíme, že operátor \kod{+} je realizován metodou \kod{\_\_add\_\_}.
%, která pracuje s instancí a dalším
%vektorem, který je obsažen v argumentu \kod{other}.
 Chceme přeci realizovat operaci \kod{x + y}, proto \kod{x}
bude instance \kod{self} a \kod{y} bude argument \kod{other.}
V této metodě provedeme příslušnou operaci sečtení dvou vektorů a vrátíme novou instanci třídy \kod{Vector},
která obsahuje výsledek.

Dalším přetíženým operátorem je \kod{\_\_str\_\_}, který nám vrací řetězec. Podrobně jako jsme používali
funkci \kod{int} nebo \kod{float} pro získání příslušné hodnoty, i funkce \kod{str} nám vrátí řetězec,
který reprezentuje daný objekt. Pro náš typ vektoru chceme, aby se nám vrátily hodnoty vektoru v kulatých
závorkách. Toho jsme docílili relativně jednoduchým trikem, kdy naše interní reprezentace vektoru je tuple,
který již odpovídá naší zvolené reprezentaci. Pro jistotu však ještě v komentáři uvádíme plnou implementaci
cílového chování metody. Tento operátor je vždy zavolán funkcí \kod{str} a tato funkce je vždy volána
při volání příkazu \kod{print}. Proto není přímé použití na první pohled plně viditelné.

Pomocí operátorů je možno značně ovlivnit chování našich typů. Na rozdíl od jiných jazyků jsme schopni
ovlivňovat i průběh volání metod nebo průběh konstrukce objektů. Toto je však již docela pokročilé téma,
které si necháme jako domácí cvičení.

\section{Kompozice}

Vytváření samotných abstraktních datových typů (tříd) bez návaznosti na okolí by asi nebylo příliš zajímavé.
Podívejme se tedy na to, jakým způsobem provádět kompozici a hlavně, kdy ji nahradit dědičností. Častokrát
se totiž stává, že se rozhodneme špatně, což pak nese neblahé následky.

Kompozici používáme pro spojení několika komponent do jednoho celku. Pokud si nejsme zcela jisti,
zdali máme použít kompozici, stačí se zeptat: \textbf{Je X součástí Y?} Pokud je odpověď kladná, jedná
se problém kompozice. Zkusme si tedy opět vylepšit náš vyhledávač reprezentovaný třídou \kod{SearchEngine}.
Tentokráte bychom chtěli vytvořit
systém, ve kterém budou obsaženy dokumenty, které jsou reprezentovány třídou \kod{Document}, kterou jsme si
již uvedli. V příkladu opět pomiňme extrémní zjednodušení, která musíme zavést.

\begin{python}
class SearchEngine(object):

    def __init__(self):
        self.documents = []

    def add_document(self, document):
        self.documents.append(document)

    def get_documents(self):
        return self.documents

    def search(self, query):
        for doc in self.documents:
            if doc.content.find(query) > 0:
                return doc

se = SearchEngine()
se.add_document(Document("Text"))
se.add_document(Document("Another text"))

for doc in se.get_documents():
    print doc.get_content()

print(se.search("text").content)
\end{python}

Na začátku jsme si řekli, že chceme modelovat systém, který obsahuje dokumenty. Dokumenty jsou tedy součástí
většího celku. Je tedy na místě využít kompozice.
V konstruktoru třídy \kod{SearchEngine} si vytvoříme instanční proměnnou typu \kod{list}, která bude
obsahovat všechny dokumenty, které bude náš vyhledávač indexovat. V metodě \kod{add\_document} pak do
interního listu přidáme referenci na předaný dokument. Metoda \kod{search} pak vyhledá patřičný dokument
podle zadaného obsahu (samozřejmě je to naivní implementace, která navíc neošetřuje chybové stavy).

Na tomto příkladu tedy vidíme, jak provést kompozici dvou komponent. Nic nám samozřejmě nebrání komponovat
více systémů do sebe. Nezapomeňme však, že je třeba dodržovat jistá pravidla, o kterých se dozvíte v dalších
předmětech jako je Softwarové inženýrství.

\section{Dědičnost}

Poslední základní vlastností OOP je dědičnost, kterou také často nazýváme specializací. Podobně jako
u kompozice, pokud si nejsme jisti, zda máme problém implementovat pomocí dědičnosti, můžeme se zeptat:
\textbf{X je Y?} Je-li odpověď kladná, jedná se o dědičnost. Častokrát také dědičnost používáme
pro převzetí kódu z nadtřídy, což je ovšem opět specializací.

Ukažme si tedy příklad, kdy chceme implementovat informační systém obsahující studenty. Entita studenta
je člověkem, tedy můžeme vytvořit nejprve obecnou třídu \kod{Person}, ze které bude
třída \kod{Student} dědit.

\begin{python}
class Person(object):

    def __init__(self, firstname, surname):
        self.firstname = firstname
        self.surname = surname

    def get_firstname(self):
        return self.firstname

    def get_surname(self):
        return self.surname

    def __str__(self):
        return " ".join((self.firstname, self.surname))


class Student(Person):

    next_id = 0

    def __init__(self, firstname, surname):
        Person.__init__(self, firstname, surname)
        Student.next_id += 1
        self.edison_id = Student.next_id

    def get_id(self):
        return self.edison_id


s1 = Student("Sean", "Connery")
print("ID:", s1.get_id(), "Student name:", s1)
# ID: 1 Student name: Sean Connery
print(s1.get_firstname(), s1.get_surname()) # Sean Connery

s2 = Student("James", "Bond")
print("ID:", s2.get_id(), "Student name:", s2)
# ID: 2 Student name: James Bond
\end{python}

Ve třídě \kod{Person} jsme naimplementovali základní funkcionalitu. Třída \kod{Student} dědí z \kod{Person}.
Navíc přidává to, že je každé instanci přiřazeno jednoznačné \kod{edison\_id}, kterým se identifikuje
v našem univerzitním IS. Také je implementována metoda \kod{get\_id}, která daný identifikátor vrací. Třída
\kod{Student} je tedy specializovanou třídou odvozenou z třídy \kod{Person}. Důležitým prvkem je
konstruktor odvozené třídy. Ten jako první vždy volá konstruktor třídy nadřazené jako statickou metodu,
kde jako instanci vloží sebe sama. Toto je jedno z mála míst, kde přímo voláme operátor, jinak se
tohoto volání snažíme vyvarovat.
\\
\\
\noindent
{\textbf{Cvičení:}}
Do třídy \kod{Person} doplňte metody pro práci s věkem člověka pomocí modulu \kod{datetime}.
Poté přetěžte takový oprátor, abyste mohli jednoduše zjistit, zda je člověk nebo student starší než nějaký
jiný. Také zkuste mezi sebou porovnat instance studenta a člověka.




%\section{Dědění vs. kompozice}

%Poucka:
%X je Y ---> X dedi z Y
%X je soucasti Y --> kompozice
%Casta chyba: X je soucasti Y --> dedicnost (Typicky spatny priklad: Usecka dedi z bodu, pritom "Usecka neni bod")

%častokrát zaměňujeme kompozici za dědičnost. Uveďme si príklad špatně zvolené implementace. Mějme





















