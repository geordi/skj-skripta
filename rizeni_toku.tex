\section{Řízení toku}

Nejznámější příkaz pro řízeni toku \kod{if}\index{if}, jsme si již uvedli v kapitole \ref{sec:syntax}.
Schématicky si jej proto pouze připomeneme:

\begin{itemize}
    \item {\color{red}if} <výraz>: <vnořený blok>
    \begin{itemize}
        \item 0+ {\color{red}elif} <výraz>: <vnořený blok>
        \item volitelně: {\color{red}else}: <vnořený blok>
    \end{itemize}
\end{itemize}

Pro vytvoření cyklů nám v jazyce Python slouží dvě konstrukce: \kod{while} a \kod{for}.

\subsection{Cyklus \kod{while}\index{while}\index{cyklus}}

Cyklus \kod{while} funguje podobně jako jsme zvyklí z jazyka C. Na začátku máme dánu podmínku iterace.
Je-li podmínka splněna, je vykonáno tělo cyklu, v opačném případě je cyklus přeskočen a pokračuje se dále
v programu. V těle cyklu můžeme použít klíčová slova \kod{break} a \kod{continue}. Slovo \kod{break}
zapříčiní ukončení cyklu. Slovo \kod{continue} zapříčiní přeskočení zbytku těla cyklu a tím pádem
vykonání dalšího cyklu za předpokladu, že je splněna vstupní podmínka.


\begin{python}
In [1]: i = 0

In [2]: while i < 5:
   ....:     print i
   ....:     i += 1
   ....:     

0
1
2
3
4
\end{python}

\subsection{Cyklus \kod{for}\index{for}\index{cyklus}} 

Cyklus \kod{for} je v jazyce Python poněkud odlišný od svého ekvivalentu v jazyce C. Jeho sémantika je podobná jako
u příkazu \kod{foreach} v jazycích Java nebo C\#. Cyklus \kod{for} je schopen procházet pouze sekvence, o kterých
jsme si něco řekli v kapitole \ref{sec:sequnces}. Podstatným rozdílem oproti cyklu \kod{for} z jazyka C je to, že
řídící proměnná cyklu postupně nabývá jednotlivých hodnot v zadané sekvenci. Odpadá tak nutnost přistupovat k prvkům
sekvence pomocí nějaké jiné proměnné. Zápis je také podstatně jednodušší. Ukažme si proto jednoduchý příklad na součet
prvků v seznamu.

%postupně v jednotlivých pruchodech

%for <name> in <iterable>:
%break, continue, else: like in while

\begin{python}
seznam = [1, 2, 5, 10, 100]

sum = 0

for prvek in seznam:
    sum += prvek

sum #118
\end{python}

I v cyklu \kod{for} můžeme použít klíčová slova \kod{break} a \kod{continue}. Ukažme si tedy příklad, kdy jsou
v seznamu uloženy i hodnoty jiného typu než \kod{int} nebo \kod{float}, a které samozřejmě spolu nemůžeme sčítat.

\begin{python}
seznam = [1.5, 2, 'Ahoj', 10, 3+5j]

sum = 0

for prvek in seznam:
    if type(prvek) == int or type(prvek) == float:
       sum += prvek

sum # 13.5
\end{python}

